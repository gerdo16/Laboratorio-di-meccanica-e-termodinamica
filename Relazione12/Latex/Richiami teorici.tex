\subsection{Legge di Snell}
La Legge di Snell (detta anche Legge della rifrazione) descrive il comportamento della luce (o di un'onda) quando passa da un mezzo a un altro con diverso indice di rifrazione, ovvero spiega quanto e come la luce cambia direzione entrando in un nuovo materiale. La legge si esprime con la formula:
\begin{equation}
	n_1\sin(\theta_1)=n_2\sin(\theta_2)
\end{equation}
dove:
\begin{itemize}
	\item $n_1$ = indice di rifrazione del primo mezzo (da cui parte il raggio),
	\item $n_2$ = indice di rifrazione del secondo mezzo (in cui entra il raggio),
	\item $\theta_1$ = angolo tra il raggio incidente e la normale alla superficie,
	\item $\theta_2$ = angolo tra il raggio rifratto e la normale.
\end{itemize}


\subsection{Funzionamento del prisma di vetro}
Un prisma ottico di vetro è un solido trasparente delimitato da due superfici piane inclinate tra loro (le facce rifrattive) che formano un angolo al vertice $\alpha$. Quando un raggio di luce entra in un prisma, subisce due rifrazioni:

\begin{enumerate}
	\item \textbf{Prima rifrazione:} alla prima faccia di ingresso, quando il raggio passa dall'aria al vetro, deviando verso la normale alla superficie.
	\item \textbf{Seconda rifrazione:} alla seconda faccia, il raggio emerge dal prisma passando dal vetro all'aria, deviando lontano dalla normale.
\end{enumerate}

Queste due deviazioni cumulative producono uno scostamento del raggio emergente rispetto alla direzione originale del raggio incidente. L'effetto complessivo si chiama \textbf{deviazione angolare}.

\begin{figure}[H]
	\centering
	\includegraphics[width=0.75\textwidth]{./figures/prismateoria}
	\caption{Indichiamo: \\
		$\theta_i$ = angolo di incidenza sulla prima faccia (rispetto alla normale) \\
		$\theta_r$ = angolo di rifrazione dentro il prisma sulla prima faccia \\
		$\theta_r'$ = angolo interno di incidenza sulla seconda faccia \\
		$\theta_i'$ = angolo di rifrazione fuori dal prisma (angolo di uscita)}
\end{figure}

Applicando la \textbf{Legge di Snell} alla prima rifrazione:
\begin{equation}
	n_{aria}\sin(\theta_i)=n_{vetro}\sin(\theta_r)
\end{equation}
Poiché $n_{aria} \approx 1$, si può semplificare:
\begin{equation}
	\sin(\theta_i)=n\sin(\theta_r)
\end{equation}
Dopo il passaggio attraverso il prisma, usando la geometria interna si trova che:
\begin{equation}
	\theta_r'=\alpha-\theta_r
\end{equation}
Applicando ancora la Legge di Snell all'uscita:
\begin{equation}
	n_{vetro}\sin(\theta_r')=n_{aria}\sin(\theta_i') \rightarrow n=\frac{\sin(\theta_i')}{\sin(\theta_r')}
\end{equation}

La deviazione totale $\delta$ è data dalla differenza angolare tra il prolungamento del raggio incidente e il raggio emergente. La formula generale è data dalla Legge (1). Se si cambia lentamente l'angolo di incidenza, l'angolo di deviazione $\delta$ dapprima diminuisce, raggiunge un valore minimo, poi ricomincia ad aumentare. In corrispondenza della deviazione minima $\delta_{min}$ accade che il percorso della luce dentro il prisma è simmetrico: $\theta_i$ e $\theta_i'$ sono uguali, così come $\theta_r$ e $\theta_r'$. In questa situazione:
\begin{equation}
	\theta_r = \theta_r'=\frac{\alpha}{2}
\end{equation}
e si può ricavare l'indice di rifrazione $n$ direttamente con una formula semplice:
\begin{equation}
	n=\frac{\sin\left(\frac{\alpha+\delta_{min}}{2}\right)}{\sin\left(\frac{\alpha}{2}\right)}
\end{equation}
e, tenendo conto dell'uguaglianza $\theta_i=\theta_i'$ e della Legge (1), otteniamo la Legge (2). La formula teorica completa per l'angolo di deviazione $\delta$ in funzione dell'angolo di incidenza $\theta_i$, per un prisma reale, viene calcolata usando la Legge di Snell all'ingresso e all'uscita del prisma e tiene conto della geometria del prisma. Si presenta nella seguente forma:

\begin{equation}
	\delta=\theta_i - \alpha + \arcsin\left(\sin(\alpha)\sqrt{n^2-\sin^2(\theta_i)}-\cos(\alpha)\sin(\theta_i)\right)
\end{equation}


\subsection{Richiami di teoria della misura}
Sia $g$ una grandezza fisica dipendente da $N$ grandezze fisiche $x_1,...,x_N$ tale che
\begin{equation}
	g=f(x_1,...,x_N)
\end{equation}
con
\begin{equation}
	x_1 = x_{1_0}\pm \Delta x_1
\end{equation}
$$ ... $$
\begin{equation}
	x_N = x_{N_0}\pm \Delta x_N
\end{equation}
La formula di propagazione dell'errore massimo è:
\begin{equation}
	\Delta g=\displaystyle\sum_{i=1}^{N}\left|\frac{\partial g}{\partial x_i}\right|_{\vec{x}=\vec{x_0}}\Delta x_i
\end{equation}
con
\begin{equation}
	\vec{x}=(x_1,...,x_N)
\end{equation}
\begin{equation}
	\vec{x_0}=(x_{1_0},...,x_{N_0})
\end{equation}
Sia g una grandezza fisica pari alla somma, o alla differenza, di N grandezze fisiche $x_1,...,x_N$ tale che
\begin{equation}
	g=x_1\pm...\pm x_N
\end{equation}
con
\begin{equation}
	x_1=x_{1_0}\pm \Delta x_N
\end{equation}
$$ ... $$
\begin{equation}
	x_N=x_{N_0}\pm \Delta x_N
\end{equation}
La formula di propagazione dell'errore massimo è:
\begin{equation}
	\Delta g= \Delta x_1+...+\Delta x_N
\end{equation}

