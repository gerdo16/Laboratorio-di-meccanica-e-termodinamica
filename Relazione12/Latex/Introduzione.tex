Scopo dell'esperienza è la determinazione dell'indice di rifrazione di un prisma di vetro mediante l'analisi del comportamento della luce al suo interno. Per raggiungere questo obiettivo, il prisma è stato posizionato su una piattaforma rotante graduata in modo tale che un raggio laser potesse inciderne una delle facce. Ruotando il prisma si osserva che l'angolo di deviazione del raggio inizialmente diminuisce, raggiunge un valore minimo e poi aumenta nuovamente, secondo la relazione:

\begin{equation}
	\delta=\theta_i+\theta_i'-\alpha
\end{equation}

Dove $\theta_i$ è l'angolo di incidenza, $\theta_i'$ è l'angolo di uscita e $\alpha$ è l'angolo compreso tra le due facce rifrattive del prisma. L'angolo di deviazione minimo è fondamentale, poiché l'angolo di incidenza ad esso associato consentirà di calcolare l'indice di rifrazione del materiale secondo la relazione:

\begin{equation}
	n=\frac{\sin(\theta_i)}{\sin(\frac{\alpha}{2})}
\end{equation}