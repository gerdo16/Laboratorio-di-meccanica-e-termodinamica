Siano assegnati due dadi, rispettivamente di quattro e venti facce. Sia inoltre $X$ la variabile aleatoria che descrive la somma dei punteggi ottenuti ad ogni lancio. Allora $X$ è una variabile discreta che assume i seguenti valori:
\begin{center}
    $\{x_i\}_{i=1,...,n} = \{2 \leq m \leq 24\}$
\end{center}
In condizioni ideali e in assenza di ulteriori informazioni, è ragionevole assumere che le coppie del tipo $(x_i, y_j)$, con $i \in \{1,...,4\}$ e $j \in \{1,...,20\}$, siano equiprobabili. Pertanto:
\begin{center}
    $P(x_i, y_i) = \frac{1}{4 \cdot 20}$, $\forall i, j$
\end{center}


%Di conseguenza per $X$ si ottiene:
%\begin{align*}
%    \{p_i\}_{i=1,...,80} = \{0.0125, 0.025, 0.0375, 0.05, 0.05, 0.05, 0.05, %\\ 0.05, 0.05, 0.05, 0.05, 0.05, 0.05, 0.05, 0.05, 0.05, 0.05, 0.05, 0.05, %0.05, 0.0375, 0.025, 0.0125\}
%\end{align*}


\subsection{Indici di Posizione: Valore Atteso}
Sia $Y$ una variabile aleatoria discreta (che per i nostri scopi possiamo assumere finita). Si definisce \textbf{valore atteso} di $Y$ la quantità:
\begin{center}
    \[\mathbb{E}[Y] = \sum_{i=1}^{n} y_i p_i\]
\end{center}
dove i valori $y_i$ rappresentano le possibili realizzazioni di $Y$ mentre $p_i$ le rispettive probabilità. Nel caso in esame, si ottiene:
\begin{center}
    $\mathbb{E}[X] = 13$
\end{center}
\subsection{Indici di Dispersione: Varianza e Deviazione Standard}
Si definisce \textbf{varianza} di $Y$ la quantità:
\begin{align*}
    \mathrm{Var}(Y) :&= \mathbb{E}[(Y-\mathbb{E}[Y])^2]= \\
    &= \sum_{i=1}^{n}p_i(y_i - \mu)^2
\end{align*}
Nel caso specifico si ottiene il seguente valore:
\begin{align*}
    \mathrm{Var}(Y) :&= 34.5
\end{align*}
Si definisce invece \textbf{deviazione standard} di $Y$ la quantità:
\begin{align*}
    \sigma := \sqrt{\text{Var}(Y)}
\end{align*}
che nel nostro caso assume il seguente valore:
\begin{align*}
    \sigma = 5.87
\end{align*}

\subsection{Considerazioni Aggiuntive}
Avendo a disposizione i valori di media $\mu$ e variazione standard $\sigma$ possiamo calcolare le probabilità che la variabile $X$ assuma valori nei seguenti intervalli:
\begin{equation*}
	[\mu-k\sigma, \mu+k\sigma]\ con\ k=1,2,3
\end{equation*} 

Otteniamo così i seguenti risultati:
\begin{itemize}
    \item $k=1: P(X \in [\mu-\sigma, \mu+\sigma]) = 0.55$
    \item $k=2: P(X \in [\mu-2\sigma, \mu+2\sigma]) = 1$
    \item $k=3: P(X \in [\mu-3\sigma, \mu+3\sigma]) = 1$
\end{itemize}

