Considerati due dadi, uno da quattro e l'altro da venti facce, e considerato come esperimento elementare il lancio di entrambi, si definisca la variabile aleatoria discreta $X$ tale da essere la somma dei valori assunti dalle facce dei due dadi. Questa variabile aleatoria discreta assumerà un valore intero compreso nell'intervallo:
\begin{equation}
	X\in[2,\ 24]
\end{equation}
Si è inizialmente determinata, mediante un istogramma a barre, la distribuzione di probabilità di tutti i possibili risultati dell'esperimento elementare definito. Si è poi calcolato il valore di aspettazione
\begin{equation}
	m=E[X]
\end{equation}
la varianza
\begin{equation}
	\sigma ^2=E[(X-m)^2]
\end{equation}
e infine le probabilità relative agli intervalli
\begin{equation}
	[m-k\sigma, m+k\sigma]\ con\ k=1,2,3
\end{equation} 
Dopodiché è stato eseguito l'esperimento elementare per un totale di 241 volte e sui dati ottenuti è stata eseguita la medesima elaborazione.