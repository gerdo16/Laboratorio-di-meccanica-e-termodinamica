Considerati due dadi, uno da quattro e l'altro da venti facce, e considerato come esperimento elementare il lancio di entrambi, si definisca la variabile aleatoria discreta $X$ tale da essere la somma dei valori assunti dalle facce dei due dadi. Questa variabile aleatoria discreta assumerà un valore intero compreso nell'intervallo:
\begin{equation}
	X\in[2,\ 24]
\end{equation}
Scopo dell'esperienza è la determinazione, mediante un istogramma a barre, della distribuzione di probabilità di tutti i possibili risultato dell'esperimento elementare definito. aa
