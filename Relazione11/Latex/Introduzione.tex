Scopo dell’esperienza è la misura del momento d’inerzia di un corpo rigido e la verifica sperimentale del teorema di Huygens-Steiner, utilizzando un sistema oscillante di tipo torsionale. L’apparato sperimentale è costituito da una molla elicoidale montata su un supporto, alla quale vengono fissati in momenti diversi un bilanciere (asta con coppia di masse come in figura x) e un disco forato (figura y). Il sistema così composto si comporta come un oscillatore armonico torsionale, il cui periodo $T$ è descritto dalla relazione:
\begin{equation}
	T=2\pi\sqrt{\frac{I}{K}}
\end{equation}
dove $I$ è il momento di inerzia del sistema rispetto all’asse di rotazione e $K$ è la costante di richiamo della molla. L'esperienza è stata articolata in 3 fasi:

\begin{itemize}
    \item \textbf{Fase preliminare}: consiste nella determinazione della costante elastica $K$. Applicando una forza $\vec{F}$ normale al bilanciere e parallela al suolo, a una distanza $r$ nota dall'asse di rotazione, viene ruotato il bilanciere di una angolo $\alpha = \pi$. il modulo del momento torcente diventa semplicemente:

    \begin{equation}
    	M = Fr
    \end{equation}
    Combinando l'equazione (2) con la seguente:

    \begin{equation}
        K=\frac{M}{\alpha}
    \end{equation}

    si riesce a ricavare una stima per $K$.
    \item \textbf{Parte I}: misura del momento d’inerzia del sistema bilanciere-masse al variare della distanza delle masse dal centro. 
    \item \textbf{Parte II}: verifica della legge di Huygens-Steiner, attraverso l’analisi del periodo di oscillazione di un disco forato, montato con assi di rotazione posti a distanze diverse dal centro di massa.
\end{itemize}