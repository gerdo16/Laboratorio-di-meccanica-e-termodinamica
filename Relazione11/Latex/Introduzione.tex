Scopo dell'esperienza è la misurazione del momento di inerzia di un corpo rigido e la verifica sperimentale del teorema di Huygens-Steiner. Per effettuare la misura, abbiamo sfruttato le proprietà di un oscillatore torsionale, costituito da una molla elicoidale su cui venivano montati, in fasi differenti dell'esperienza, un bilanciere con masse e un disco forato. Una fase preliminare fondamentale ha riguardato la determinazione della costante di richiamo $K$ della molla. A tale scopo, è stata applicata una forza tangenziale al bilanciere tramite un dinamometro, posizionando il punto di applicazione a una distanza nota dal centro di rotazione. Misurando la forza e conoscendo l'angolo di rotazione imposto (pari a $180^{\circ}$, ovvero $\pi \ rad$), è stato possibile calcolare il momento torcente applicato e quindi risalire alla costante elastica della molla:
\begin{equation}
	M = Fr
\end{equation}
\begin{equation}
	K=\frac{M}{\alpha}
\end{equation}
Questo valore è stato poi impiegato nelle successive analisi per la determinazione del momento di inerzia. L'esperienza si è quindi articolata in due fasi principali: la prima dedicata alla misura del momento di inerzia del bilanciere al variare della posizione delle masse; la seconda alla verifica sperimentale del teorema di Huygens-Steiner, attraverso l'analisi delle oscillazioni di un disco forato in diverse posizioni. L'oscillatore è stato posto in oscillazione, e il periodo è stato misurato con un cronometro. Grazie alla relazione:
\begin{equation}
	T=2\pi\sqrt{\frac{I}{K}}
\end{equation}
è stato possibile determinare il momento d'inerzia.