

\subsection{Parte I} Avendo disposto due masse approssimativamente uguali $m_1 = m_2 = m$ in posizione simmetrica 
rispetto all'asse di rotazione, l'equazione (4) diventa semplicemente:

\begin{equation}
    I = 2mr^2
\end{equation}

dove $r$ rappresenta la semidistanza tra le due masse 
(o equivalentemente, la distanza di ciascuna massa dall'asse di rotazione). 
Ci attendiamo pertanto una dipendenza lineare del momento d'inerzia $I$ in 
funzione del quadrato della distanza $d^2$. Come mostra il grafico in Figura 1, 
le previsioni sono confermate dai dati sperimentali in maniera più che soddisfacente.

\subsection{Parte II}

Dall'equazione (36), ci attendiamo una dipendenza lineare di $I$ in funzione di $r^2$. 
Il grafico in Figura 2 conferma approssimativamente le previsioni, 
anche se in maniera meno accurata rispetto a quanto fatto nella parte I. 
La causa principale risiede nella difficoltà incontrata nel realizzare le stesse condizioni 
ad ogni misurazione del periodo di oscillazione del disco, per la mancanza di riferimenti 
stabili rispetto a cui riportare il disco ad ogni misurazione.