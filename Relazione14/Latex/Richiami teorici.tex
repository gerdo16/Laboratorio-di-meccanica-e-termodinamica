Per stimare \( \tau \) a partire dalla (1) si può procedere con il metodo di \textbf{linearizzazione}. Applicando il logaritmo naturale alla legge infatti, si ottiene:

\begin{align}
    \ln\left( T(t) - T_2 \right) &= \ln [(T_1-T_2)e^{-t/\tau}] = \\
    &= \ln(T_1-T_2) + \ln e^{-t/\tau} = \\
    &= \ln\left( T_1 - T_2 \right) - \frac{t}{\tau}
\end{align}

Questa relazione suggerisce che, rappresentando \( \ln\left( T(t) - T_2 \right) \) in funzione di \( t \), i dati dovrebbero disporsi lungo una retta con:
\begin{itemize}
    \item \textbf{coefficiente angolare} pari a \( -\frac{1}{\tau} \),
    \item \textbf{intercetta} pari a \( \ln\left( T_1 - T_2 \right) \).
\end{itemize}
In questo modo, tramite un semplice fit lineare, è possibile stimare il valore di \( \tau \).
\subsection{Richiami Statistici}
Il metodo dei minimi quadrati è una tecnica che permette di trovare una funzione, rappresentata da una curva di regressione, che si avvicini il più possibile ad un insieme di dati (tipicamente punti del piano). In particolare, la funzione trovata deve essere quella che minimizza la somma dei quadrati delle distanze tra i dati osservati e quelli della curva che rappresenta la funzione stessa. Siano $b$ il coefficiente angolare e $a$ l'intercetta della retta di regressione:
\begin{equation}
	b=\frac{\displaystyle\sum_{i=1}^{N}[(x_i-\overline{x})(y_i-\overline{y})]}{\displaystyle\sum_{i=1}^{N}(x_i-\overline{x})^2}
\end{equation}
\begin{equation}
	a=\overline{y}-b\overline{x}
\end{equation}
Con $\overline{x}=\frac{\displaystyle\sum_{i=1}^{N}x_i}{N}$ e $\overline{y}=\frac{\displaystyle\sum_{i=1}^{N}y_i}{N}$
mentre le incertezze:
\begin{equation}
	\Delta b=3\sigma_b
\end{equation}
\begin{equation}
	\Delta a=3\sigma_a
\end{equation}
Con $$\sigma_b=\sigma_y\sqrt{\frac{N}{\Delta}}$$
$$\sigma_y=\sqrt{\frac{\displaystyle\sum_{i=1}^{N}(y_i-bx_i-a)^2}{N-2}}$$
$$\sigma_a=\sigma_y\sqrt{\frac{\displaystyle\sum_{i=1}^{N}x_i^2}{\Delta}}$$
$$\Delta=N\displaystyle\sum_{i=1}^{N}(x_i-\overline{x})^2$$

\subsection{Richiami di teoria della misura}
Sia $g$ una grandezza fisica dipendente da $N$ grandezze fisiche $x_1,...,x_N$ tale che
\begin{equation}
	g=f(x_1,...,x_N)
\end{equation}
con
\begin{equation}
	x_1 = x_{1_0}\pm \Delta x_1
\end{equation}
$$ ... $$
\begin{equation}
	x_N = x_{N_0}\pm \Delta x_N
\end{equation}
La formula di propagazione dell'errore massimo è:
\begin{equation}
	\Delta g=\displaystyle\sum_{i=1}^{N}\left|\frac{\partial g}{\partial x_i}\right|_{\vec{x}=\vec{x_0}}\Delta x_i
\end{equation}
con
\begin{equation}
	\vec{x}=(x_1,...,x_N)
\end{equation}
\begin{equation}
	\vec{x_0}=(x_{1_0},...,x_{N_0})
\end{equation}
Sia g una grandezza fisica pari alla somma, o alla differenza, di N grandezze fisiche $x_1,...,x_N$ tale che
\begin{equation}
	g=x_1\pm...\pm x_N
\end{equation}
con
\begin{equation}
	x_1=x_{1_0}\pm \Delta x_N
\end{equation}
$$ ... $$
\begin{equation}
	x_N=x_{N_0}\pm \Delta x_N
\end{equation}
La formula di propagazione dell'errore massimo è:
\begin{equation}
	\Delta g= \Delta x_1+...+\Delta x_N
\end{equation}


