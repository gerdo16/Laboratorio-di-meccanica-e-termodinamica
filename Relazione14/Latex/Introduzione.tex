Scopo dell'esperienza è determinare la \textbf{costante di tempo} \( \tau \) di un termometro, 
un parametro che descrive la velocità con cui il termometro risponde a variazioni termiche. 
Per farlo, il termometro è stato trasferito da un bagno a temperatura $T_1$ ad un altro a temperatura $T_2$. All'equilibrio, il termometro raggiungerà la temperatura $T_2$, tuttavia il passaggio tra le due temperature non avviene istantaneamente ma attraverso una fase transiente in cui la temperatura si porta gradualmente a $T_2$. La legge che descrive questo processo è la seguente:
\begin{equation}
    T(t) = T_2 + (T_1 - T_2) e^{-t/\tau}
\end{equation}
dove:
\begin{itemize}
    \item \( T(t) \) è la temperatura letta dal termometro all'istante \( t \),
    \item \( T_1 \) è la temperatura iniziale del termometro,
    \item \( T_2 \) è la temperatura finale dell’ambiente in cui viene immerso,
    \item \( \tau \) è la costante di tempo del termometro.
\end{itemize}
La \textbf{costante di tempo} \( \tau \) rappresenta l'intervallo di tempo necessario affinché la differenza tra la temperatura del termometro e la temperatura finale si riduca di un fattore \( e \) (circa 2,718). Più il valore di \( \tau \) è piccolo, più rapidamente il termometro raggiunge l'equilibrio termico. Utilizzando due becher ed un fornellino in dotazione, sono stati preparati due bagni, uno “freddo” con acqua a temperatura ambiente, che indichiamo con $T_1$ l’altro “caldo” con acqua in ebollizione, la cui temperatura indichiamo con $T_2$. La temperatura dell’acqua del bagno “freddo” è stata misurata con un termometro digitale. Inizialmente, il termometro è stato immerso nel bagno caldo e lasciato fino al raggiungimento della temperatura di equilibrio $T_2$. Successivamente, è stato immerso nel bagno freddo per essere raffreddato, portandosi ad una temperatura intermedia di equilibrio $T'$:
\begin{equation}
    T_1 > T' > T_2
\end{equation}
La procedura è stata poi ripetuta per 3 volte. Ciascun ciclo di misurazioni è stato ripetuto per un totale di 8 volte.
