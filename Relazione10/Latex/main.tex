\documentclass[11pt]{article}

\usepackage[utf8]{inputenc}
\usepackage[margin=1in]{geometry} 
\usepackage{amsmath,amsthm,amssymb,graphicx,mathtools,tikz,hyperref,multicol,cancel,enumitem,booktabs,float,pgfplots,multirow,mathrsfs,textcomp,gensymb,soul,changepage,threeparttable}
%\usepackage[table]{xcolor}
\usepackage[T1]{fontenc}
\usepackage[italian]{babel}
\usepackage{hyphenat}
\hyphenation{mate-mati-ca recu-perare}
\usetikzlibrary{positioning}
\pgfplotsset{compat=1.14}

\newcommand{\n}{\mathbb{N}}
\newcommand{\z}{\mathbb{Z}}
\newcommand{\q}{\mathbb{Q}}
\newcommand{\cx}{\mathbb{C}}
\newcommand{\real}{\mathbb{R}}
\newcommand{\field}{\mathbb{F}}
\newcommand{\ita}[1]{\textit{#1}}
\newcommand{\com}[2]{#1\backslash#2}
\newcommand{\oneton}{\{1,2,3,...,n\}}
\newcommand{\idea}[1]{\begin{gather*}#1\end{gather*}}
\newcommand{\ef}{\ita{f} }
\newcommand{\eff}{\ita{f}}
\newcommand{\proofs}[1]{\begin{proof}#1\end{proof}}
\newcommand{\inv}[1]{#1^{-1}}
\newcommand{\setb}[1]{\{#1\}}
\newcommand{\en}{\ita{n }}
\newcommand{\vbrack}[1]{\langle #1\rangle}
\newcommand{\qRa}{\quad \Rightarrow \quad}
\newcommand{\smaca}[1]{\textbf{\textsc{#1}}}

\newenvironment{theorem}[2][Teorema]{\begin{trivlist}
		\item[\hskip \labelsep {\bfseries #1}\hskip \labelsep {\bfseries #2.}]}{\end{trivlist}}
\newenvironment{lemma}[2][Lemma]{\begin{trivlist}
		\item[\hskip \labelsep {\bfseries #1}\hskip \labelsep {\bfseries #2.}]}{\end{trivlist}}
\newenvironment{exercise}[2][Esercizio]{\begin{trivlist}
		\item[\hskip \labelsep {\bfseries #1}\hskip \labelsep {\bfseries #2.}]}{\end{trivlist}}
\newenvironment{proposition}[2][Proposizione]{\begin{trivlist}
		\item[\hskip \labelsep {\bfseries #1}\hskip \labelsep {\bfseries #2.}]}{\end{trivlist}}
\newenvironment{corollary}[2][Corollario]{\begin{trivlist}
		\item[\hskip \labelsep {\bfseries #1}\hskip \labelsep {\bfseries #2.}]}{\end{trivlist}}

\hypersetup {
	colorlinks,
	linkcolor=blue
}

\graphicspath{{img/}}

\begin{document}
	\setlength{\parindent}{0pt}
	\title{\vspace{-4em}{\large Laboratorio di Meccanica e Termodinamica} \\
		Relazione di Laboratorio}
	\author{Gruppo 3 \\
		Gerardo Selce, Maurizio Liguori, Emanuela Galluccio}
	\date{01/04/2025}
	\maketitle
	
	\vspace{-2em}\par\noindent\rule{\textwidth}{0.4pt}
	\begin{center}
		{\Large\sc Calcolo del coefficiente di rifrazione dell'acqua}
	\end{center}
	\par\noindent\rule{\textwidth}{0.4pt}
	
	
	\section{Introduzione}
	Considerati due dadi, uno da quattro e l'altro da venti facce, e considerato come esperimento elementare il lancio di entrambi, si definisca la variabile aleatoria discreta $X$ tale da essere la somma dei valori assunti dalle facce dei due dadi. Questa variabile aleatoria discreta assumerà un valore intero compreso nell'intervallo:
\begin{equation}
	X\in[2,\ 24]
\end{equation}
Si è inizialmente determinata, mediante un istogramma a barre, la distribuzione di probabilità di tutti i possibili risultati dell'esperimento elementare definito. Si è poi calcolato il valore di aspettazione
\begin{equation}
	m=E[X]
\end{equation}
la varianza
\begin{equation}
	\sigma ^2=E[(X-m)^2]
\end{equation}
e infine le probabilità relative agli intervalli
\begin{equation}
	[m-k\sigma, m+k\sigma]\ con\ k=1,2,3
\end{equation} 
Dopodiché è stato eseguito l'esperimento elementare per un totale di 241 volte e sui dati ottenuti è stata eseguita la medesima elaborazione.
	
	\section{Richiami teorici}
	\section{Introduction}
L'ottica geometrica è una branca della fisica classica che permette di descrivere 
efficacemente la propagazione della luce, nell'approssimazione in cui la lunghezza d'onda $\lambda$ del raggio luminoso 
sia molto inferiore alle dimensioni $d$ degli ostacoli incontrati: 
\begin{center}
    $\lambda << d$
\end{center}
In tale contesto, la luce può essere rappresentata come un fascio di raggi rettilinei, 
ciascuno dei quali rappresenta la direzione di propagazione dell'onda luminosa. Si tratta di un modello semplificato che 
permette però di descrivere un'ampia gamma di fenomeni in maniera soddisfacente e senza ricorrere al concetto di onda. 
\paragraph{}
In un mezzo omogeneo e trasparente, la luce si propaga secondo traiettorie rettilinee. 
L'evidenza sperimentale mostra infatti che  un corpo opaco interposto tra una sorgente luminosa puntiforme e uno schermo, 
proietta un'ombra con contorni ben definiti, la cui forma dipende dalla geometria dell'ostacolo e dalla direzione dei raggi incidenti, 
compatibilmente con il principio secondo cui i raggi luminosi si propagano in linea retta e non possono aggirare gli ostacoli.
\begin{figure}[H]
  \centering
  \includegraphics[width=0.55\textwidth]{../figures/pendolo.png}
  \caption{La regione d’ombra, al di là
dell’ostacolo è limitata al solo
cono avente per vertice la sorgente
puntiforme S e generatrici tangenti
all’ostacolo.}
\end{figure}

Questo principio trova applicazione pratica nella camera oscura, in cui un piccolo foro proietta un'immagine invertita della sorgente 
luminosa su uno schermo opposto.
\begin{figure}[H]
  \centering
  \includegraphics[width=0.55\textwidth]{../figures/camera_oscura.png}
  \caption{Camera Oscura}
\end{figure}

Quando un raggio luminoso attraversa l'interfaccia tra due mezzi con diverso indice di rifrazione, 
subisce una variazione di direzione. La rifrazione è descritta dalle seguenti leggi:
Il raggio incidente, il raggio rifratto e la normale alla superficie nel punto di incidenza giacciono nello stesso piano.
La relazione tra gli angoli di incidenza e di rifrazione è espressa dalla legge di Snell:



dove x  e y sono gli indici di rifrazione dei due mezzi, z è l'angolo di incidenza e v è l'angolo di rifrazione.
L'indice di rifrazione  è definito come il rapporto tra la velocità della luce nel vuoto  
e la velocità della luce nel mezzo considerato :



\subsection{Richiami Matematici}
Il metodo dei minimi quadrati è una tecnica che permette di trovare una funzione, rappresentata da una curva di regressione, che si avvicini il più possibile ad un insieme di dati (tipicamente punti del piano). In particolare, la funzione trovata deve essere quella che minimizza la somma dei quadrati delle distanze tra i dati osservati e quelli della curva che rappresenta la funzione stessa. Siano $b$ il coefficiente angolare e $a$ l'intercetta della retta di regressione:
\begin{equation}
    b=\frac{\displaystyle\sum_{i=1}^{N}[(x_i-\overline{x})(y_i-\overline{y})]}{\displaystyle\sum_{i=1}^{N}(x_i-\overline{x})^2}
\end{equation}
\begin{equation}
    a=\overline{y}-b\overline{x}
\end{equation}
Con $\overline{x}=\frac{\displaystyle\sum_{i=1}^{N}x_i}{N}$ e $\overline{y}=\frac{\displaystyle\sum_{i=1}^{N}y_i}{N}$
mentre le incertezze:
\begin{equation}
    \Delta b=3\sigma_b
\end{equation}
\begin{equation}
    \Delta a=3\sigma_a
\end{equation}
Con $$\sigma_b=\sigma_y\sqrt{\frac{N}{\Delta}}$$
$$\sigma_y=\sqrt{\frac{\displaystyle\sum_{i=1}^{N}(y_i-bx_i-a)^2}{N-2}}$$
$$\sigma_a=\sigma_y\sqrt{\frac{\displaystyle\sum_{i=1}^{N}x_i^2}{\Delta}}$$
$$\Delta=N\displaystyle\sum_{i=1}^{N}(x_i-\overline{x})^2$$

\subsection{Richiami di teoria della misura}
Sia $g$ una grandezza fisica dipendente da $N$ grandezze fisiche $x_1,...,x_N$ tale che
\begin{equation}
    g=f(x_1,...,x_N)
\end{equation}
con
\begin{equation}
    x_1 = x_{1_0}\pm \Delta x_1
\end{equation}
$$ ... $$
\begin{equation}
    x_N = x_{N_0}\pm \Delta x_N
\end{equation}
La formula di propagazione dell'errore massimo è:
\begin{equation}
    \Delta g=\displaystyle\sum_{i=1}^{N}\left|\frac{\partial g}{\partial x_i}\right|_{\vec{x}=\vec{x_0}}\Delta x_i
\end{equation}
con
\begin{equation}
    \vec{x}=(x_1,...,x_N)
\end{equation}
\begin{equation}
    \vec{x_0}=(x_{1_0},...,x_{N_0})
\end{equation}


	
	\section{Apparato sperimentale}
	Per svolgere questa esperienza è stato utilizzato il seguente apparato sperimentale:
\begin{itemize}
	\item Dado a 4 facce
	\item Dado a 20 facce
\end{itemize}


	
	\section{Descrizione e analisi dei dati sperimentali}
	\subsection{Operazioni Preliminari}
Per prima cosa si procede al montaggio dell'apparato sperimentale, formato da un bilanciere collegato ad un supporto con molla, come in figura (x). Successivamente, si procede a ruotare il bilanciere di 180\(^\circ\) rispetto alla posizione di equilibrio. Disponendo un dinamometro ortogonalmente al bilanciere e parallelamente al suolo, è stato possibile misurare la forza di richiamo della molla in corrispondenza di diversi valori della distanza $r$ del punto di applicazione del dinamometro dal centro del bilanciere. La disposizione del dinamometro ci consente di calcolare il modulo del momento della forza di richiamo semplicemente come 
\begin{equation}
    M = F \cdot r
\end{equation}
I dati relativi alle distanze e alle forze di richiamo misurate sono contenuti nelle seguenti tabelle:

\begin{table}[H]
	\centering
	\begin{tabular}{|c|c|}
		\hline
		& \textbf{r $[m]$ } \\
		\hline
		  $r_1$ & $0.3005\pm 0.0005$ \\
		$r_2$ & $0.2505\pm 0.0005$ \\
		$r_3$ & $0.2005\pm 0.0005$ \\
		$r_4$ & $0.1505\pm 0.0005$ \\
		$r_5$ & $0.1005\pm 0.0005$ \\
            $r_6$ & $0.0505 \pm 0.0005$ \\
		\hline
	\end{tabular}
	\caption{Distanze $r$ del punto di applicazione del dinamometro dal centro del bilanciere.}
	\label{tab:}
\end{table}

\begin{table}[H]
	\centering
	\begin{tabular}{|c|}
		\hline
		\textbf{F $[N]$} \\
		\hline
		$0.175\pm 0.025$ \\
		$0.225\pm 0.025$ \\
		$0.275\pm 0.025$ \\
		$0.375\pm 0.025$ \\
		$0.550\pm 0.025$ \\
            $1.10 \pm 0.03$ \\
		\hline
	\end{tabular}
	\caption{Forze di richiamo $F$ al variare della distanza $r$.}
	\label{tab:}
\end{table}

In corrispondenza dei valori di $r$ ed $F$ misurati, si ottengono i valori del momento $M$:

\begin{table}[H]
	\centering
	\begin{tabular}{|c|}
		\hline
		\textbf{M $[N*m]$} \\
		\hline
		$0.0525\pm 0.0004$ \\
		$0.0563\pm 0.0004$ \\
		$0.0550\pm 0.0003$ \\
		$0.0563\pm 0.0003$ \\
		$0.0550\pm 0.0003$ \\
            $0.0550 \pm 0.0001$ \\
		\hline
	\end{tabular}
	\caption{Momento $M$ della forza di richiamo al variare della distanza $r$.}
	\label{tab:}
\end{table}

I valori di $M$ sono stati calcolati utilizzando la formula (24), arrotondandoli al numero minimo di cifre significative tra quelle dei fattori $r$ ed $F$, pari a tre. Le incertezze su $M$ sono state ottenute al seguente modo:
\begin{itemize}
    \item Si calcola l'incertezza relativa con la seguente formula di propagazione: 
    \begin{equation}
        \frac{\delta M}{|M|} = \frac{\delta F}{|F|} + \frac{\delta r}{|r|}
    \end{equation}
    \item Si calcola l'incertezza assoluta:
    \begin{equation}
        \delta M = \frac{\delta M}{|M|} \cdot |M|
    \end{equation}
\end{itemize}

Possiamo infine calcolare la migliore stima di $M$ come media dei valori riportati in tabella (4) e l'incertezza associata come deviazione standard della media:

\begin{itemize}
    \item Stima migliore per $M$:
    \begin{equation}
        M_{best} = \frac{1}{N}\sum_{i=1}^{N} x_i
    \end{equation}
    dove $N$ è il numero di misurazioni disponibili, $x_i$ i rispettivi valori. Nel nostro caso,
    \begin{equation}
        M_{best} = 0.055 \ Nm
    \end{equation}
    \item Deviazione standard della media:
    \begin{equation}
        \sigma_{\bar{x}} = \sigma_x/\sqrt{N}
    \end{equation}
    ove $\bar{x} = M_{best}$ e $\sigma_x$ indica la deviazione standard dei valori di $M$. Nel nostro caso:
    \begin{equation}
        \sigma_{\bar{x}} = 0.001 \ Nm
    \end{equation}
\end{itemize}
A questo punto, è possibile sfruttare l'equazione (6) per stimare il modulo di $K$:
\begin{equation}
    K = \frac{M}{\pi} 
\end{equation}
otteniamo così la seguente conclusione: 
\begin{itemize}
    \item $K = 0.018 \ \frac{Nm}{rad}$. 
    \item $\delta K = \frac{1}{\pi}\cdot \delta M = 0.001 \  \frac{Nm}{rad}$
\end{itemize}

\subsection{Parte I: Misura del momento d'inerzia di un corpo rigido}
Scopo della presente sezione è la misurazione del momento d'inerzia di una coppia di masse approssimativamente uguali, disposte simmetricamente rispetto all'asse perpendicolare al bilanciere e passante per il suo centro. Dopo aver posizionato le masse sul bilanciere in posizione simmetrica, si procede a determinare e registrare la posizione di equilibrio del sistema. Ruotando il bilanciere di un angolo pari a $\frac{\pi}{2}$ rispetto alla posizione di equilibrio, il sistema viene lasciato in oscillazione libera. Si procede così alla misurazione del tempo corrispondente a 5 oscillazioni a partire dal primo passaggio dalla posizione di equilibrio. La misura è stata ripetuta 4 volte alternando le rotazioni in senso orario e antiorario, in maniera tale da poter ricavare una stima accettabile del periodo di oscillazione. La procedura è stata ripetuta 6 volte, riducendo ad ogni passaggio la distanza delle due masse dal centro secondo quanto riportato in tabella (2).

\begin{table}[H]
	\centering
	\begin{tabular}{|c|c|c|c|c|c|}
		\hline
            Distanze & $T_1$ [s] & $T_2$ [s]& $T_3$ [s]& $T_4$ [s] & $\delta T$ [s]\\
            \hline
            $r_1$ & 47.92 & 47.75 & 47.72 & 47.72 & 0.01\\
            \hline
            $r_2$ & 40.40 & 40.44 & 40.50 & 40.44 & 0.01\\
            \hline
            $r_3$ & 33.77 & 33.65 & 33.58 & 33.60 & 0.01\\
            \hline
            $r_4$ & 26.87 & 26.87 & 26.77 & 26.70 & 0.01\\
            \hline
            $r_5$ & 20.73 & 20.77 & 20.77 & 20.75 & 0.01\\
            \hline
            $r_6$ & 16.00 & 16.14 & 16.16 & 16.05 & 0.01\\
            \hline
        \end{tabular}
	\caption{Tabella dei periodi in funzione delle distanze dal centro del bilanciere.}
	\label{tab:}
\end{table}

Poichè per ogni distanza abbiamo a disposizione solo poche misure del periodo $T$ che si ripetono, possiamo stimare questo valore al seguente modo:
\begin{itemize}
    \item Stima del periodo:
    \begin{equation}
        T_{best} = \frac{T_{max} + T_{min}}{2}
    \end{equation}
    dove $T_{max}$ è il periodo più alto e $T_{min}$ quello più basso rispetto ad una distanza fissata.
    \item Stima dell'incertezza con il metodo della semidispersione: 
    \begin{equation}
        \delta T = \frac{T_{max} - T_{min}}{2}
    \end{equation}
\end{itemize}

La tabella che segue riporta i valori numerici delle stime ottenute tenendo conto delle (32) e (33):

\begin{table}[H]
	\centering
	\begin{tabular}{|c|}
		\hline
		\textbf{I $[kg \cdot m^2]$} \\
		\hline
		$47.82\pm 0.10$ \\
		$40.45\pm 0.05$ \\
		$33.68\pm 0.10$ \\
		$26.79\pm 0.09$ \\
		$20.75\pm 0.02$ \\
            $16.08 \pm 0.08$ \\
		\hline
	\end{tabular}
	\caption{Periodi di oscillazione.}
	\label{tab:}
\end{table}


A questo punto, mediante semplici calcoli, è possibile ricavare una stima del momento d'inerzia dalla formula (3):
\begin{equation}
    I = \frac{K \cdot T^2}{4\pi^2}
\end{equation} 
Osserviamo che i valori ricavati per $I$ misurano il momento d'inerzia del sistema bilanciere-masse e pertanto contengono un termine addizionale rappresentato dal momento d'inerzia del bilanciere vuoto $I_a = 0.01 \pm 0.01 \ kg \cdot m^2$, che va sottratto dai valori ottenuti dalla (34), per ottenere il momento d'inerzia delle sole masse. La propagazione dell'incertezza su $I$ è stata valutata con il metodo dei differenziali:

\begin{equation}
    \delta I = \frac{T^2 \delta K + 2 KT \delta T}{4\pi^2} + \delta I_a
\end{equation}

La tabella che segue riporta i valori numerici delle stime ottenute tenendo conto delle (34) e (35):

\begin{table}[H]
	\centering
	\begin{tabular}{|c|}
		\hline
		\textbf{I $[kg \cdot m^2]$} \\
		\hline
		$1.0\pm 0.2$ \\
		$0.72\pm 0.01$ \\
		$0.50\pm 0.01$ \\
		$0.31\pm 0.01$ \\
		$0.19\pm 0.01$ \\
            $0.11 \pm 0.01$ \\
		\hline
	\end{tabular}
	\caption{Momento d'inerzia al variare della distanza dal centro del bilanciere.}
	\label{tab:}
\end{table}

Poichè $K$ e $T$ sono riportati rispettivamente con 2 e 4 cifre significative, il momento di inerzia $I$ è stato riportato con 2 cifre significative.

Di seguito viene riportato il grafico di $I$ in funzione di $d^2$.

\begin{figure}[h]
    \centering
    \includegraphics[width=0.6\textwidth]{figures/graph_1.png}
    \caption{Grafico del momento d'inerzia in funzione del quadrato della distanza}
    \label{fig:etichetta}
\end{figure}

\subsection{Parte II: Verifica Del Teorema di Huygens-Steiner}
Scopo della presente sezione è verificare che il momento d'inerzia di un disco rispetto ad un asse parallelo ad un asse baricentrico e a distanza $d$ da quest'ultimo dipende linearmente da $d^2$, coerentemente con il teorema di Huygens-Steiner il cui enunciato è riportato di seguito:

\begin{teorema}[Huygens-Steiner]
Il momento d’inerzia di un corpo rigido rispetto a un asse qualsiasi parallelo a un asse passante per il baricentro è dato dalla somma del momento d’inerzia rispetto all’asse baricentrico G e del prodotto tra la massa e il quadrato della distanza tra i due assi: 
\[
I = I_G + m d^2
\]
\end{teorema}

Un disco forato è stato montato al supporto con molla, utilizzando preliminarmente il foro centrale. Ruotando il disco di un angolo $\frac{\pi}{2}$ rispetto alla posizione di equilibrio, si procede con la misurazione del periodo corrispondente a 5 oscillazioni, contate azionando il cronometro in corrispondenza del primo passaggio dalla posizione di equilibrio fissata precedentemente. La misura è stata ripetuta per 4 volte, ruotando alternativamente il disco in senso orario e antiorario. La procedura appena descritta è stata ripetuta variando la posizione del disco utilizzando i fori praticati sulla sua superficie. Le distanze utilizzate sono riportate nella seguente tabella:

\begin{table}[H]
	\centering
	\begin{tabular}{|c|}
		\hline
		\textbf{d $[m]$} \\
		\hline
		$0.0000\pm 0.0005$ \\
		$0.0205\pm 0.0005$ \\
		$0.0405\pm 0.0005$ \\
		$0.0605\pm 0.0005$ \\
		$0.0805\pm 0.0005$ \\
            $0.101 \pm 0.001$ \\
		\hline
	\end{tabular}
	\caption{Distanze dell'asse rispetto all'asse baricentrico ortogonale al disco.}
	\label{tab:}
\end{table}

Il momento d'inerzia del disco rispetto agli assi corrispondenti alle distanze riportate in tabella è stato calcolando seguendo la stessa procedura della prima parte, ottenendo i seguenti risultati:

\begin{table}[H]
	\centering
	\begin{tabular}{|c|}
		\hline
		\textbf{I $[kg \cdot m^2]$} \\
		\hline
		$0.291\pm 0.006$ \\
		$0.316\pm 0.005$ \\
		$0.319\pm 0.004$ \\
		$0.352\pm 0.005$ \\
		$0.472\pm 0.017$ \\
            $0.535 \pm 0.009$ \\
		\hline
	\end{tabular}
	\caption{Momenti d'inerzia}
	\label{tab:}
\end{table}

Di seguito, riportiamo i valori del momento d'inerzia $I$ in funzione del quadrato della distanza $d^2$ fra il foro centrale e quello utilizzato per il fissaggio del disco. 

\begin{figure}[H]
    \centering
    \includegraphics[width=0.6\textwidth]{figures/graph_2.png}
    \caption{Grafico del momento d'inerzia in funzione del quadrato della distanza}
    \label{fig:etichetta}
\end{figure}
	
	\section{Conclusioni}
	Il valore di fabbrica dichiarato della distanza focale è pari a:
\begin{equation}
	f_{reale} = 15\ cm
\end{equation}

Dalla retta di regressione in Figura (3) ricaviamo dal valore dell'intercetta:
\begin{equation}
	f = \frac{1}{a} = 18.18 cm
\end{equation}
e la relativa incertezza come riportato nella Legge (25):
\begin{equation}
	\Delta f = \frac{1}{a^2}\Delta a = 0.27 cm
\end{equation}

Notiamo che la stima della distanza focale si avvicina a quella dichiarata con una discrepanza 
di circa $3\ cm$, imputabile alla presenza di errori sistematici nella misurazione delle distanze $p$ e $q$ dell'apparato sperimentale. 

	
\end{document}