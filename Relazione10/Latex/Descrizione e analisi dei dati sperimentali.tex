Per svolgere questo esperimento sono stati innanzitutto scelti dei riferimenti:
\begin{itemize}
	\item La distanza tra la vaschetta e l'asta $d=(57.50\pm0.05)\ cm$.
	\item L'altezza dell'asta dalla quale sono state effettuate le osservazioni $D_0=(42.35\pm0.05)\ cm$.
\end{itemize}
Dopodiché è stata misurata l'altezza della vaschetta, ottenendo il seguente risultato: $D=(11.85\pm0.05)\ cm$. Queste tre grandezze sono state misurate utilizzando la riga. Per ognuna di loro sono state effettuate tre misurazioni e, poiché la loro semidispersione è minore dell'errore strumentale, la loro incertezza corrisponde proprio a quest'ultimo. Con queste tre misure è stato possibile stabilire, utilizzando la carta millimetrata posta sul fondo della vaschetta, il punto $O=(19.50\pm0.05) cm$ (vedi Figura(3)).