Scopo dell'esperienza è la stima del fattore di conversione tra joule e calorie, ovvero quanta energia in joule corrisponde a una caloria. Per farlo è stato utilizzato il calorimetro di Regnault. Nel calorimetro è stata versata una quantità d'acqua $M$ e al suo coperchio sono stati fissati due resistori. Una volta chiuso il calorimetro, tramite dei cavi, i due resistori sono stati posti in serie e connessi ad un generatore di tensione. In questo modo la corrente circolante, per effetto Joule, riscalda i due resistori e il calore si conduce all'acqua circostante, aumentandone la temperatura. L'energia dissipata dai due resistori $E$ e la quantità di calore assorbito dal sistema $acqua+calorimetro$ sono legati dalla relazione:

\begin{equation}
	k = \frac{E}{Q}
\end{equation}

E $k$ rappresenta il fattore di conversione $\frac{joule}{caloria}$ la cui stima reale è:

\begin{equation}
	k = 4.186 \frac{J}{cal}
\end{equation}

