\subsection{Fase 1}
È stata inizialmente misurata la massa $m_1$ d'acqua con la bilancia digitale. Sono state raccolte tre misurazioni. Il valore finale è stato stimato come la media delle tre misurazioni e l'incertezza è quella strumentale poiché maggiore della semidispersione.
\begin{equation}
	m_1=(140.45\pm 0.01)\ g
\end{equation}
Una volta versata la quantità d'acqua $m_1$ all'interno del calorimetro ne è stata misurata la temperatura $T_1$ utilizzando il termometro digitale.
\begin{equation}
	T_1=(19.1\pm 0.1)\ ^{\circ}
\end{equation}
È stata poi stimata la massa d'acqua $m_2$ con lo stesso procedimento utilizzato per la massa $m_1$.
\begin{equation}
	m_2=(111.57\pm 0.01)\ g
\end{equation}
Una volta versata la quantità d'acqua $m_2$ all'interno di un becher, quest'ultimo è stato posto su un fornello elettrico e fatto riscaldare, portando l'acqua alla temperatura $T_2$.
\begin{equation}
	T_2=(94.2\pm 0.1)\ ^{\circ}
\end{equation}
A questo punto la quantità d'acqua $m_2$ è stata versata rapidamente nel calorimetro ed è stata misurata la temperatura di equilibrio $T_e$.
\begin{equation}
	T_e=(49.5\pm 0.1)\ ^{\circ}
\end{equation}
Con queste misurazioni dirette è possibile stimare l'equivalente in massa del calorimetro $m_e$. La sua incertezza è stata valutata col metodo dei differenziali (Legge (17)).
\begin{equation}
	\Delta m_e=\left|\frac{\partial m_e}{\partial m_2} \right|\Delta m_2+\left|\frac{\partial m_e}{\partial m_1} \right|\Delta m_1 + \left|\frac{\partial m_e}{\partial T_2} \right|\Delta T_2 + \left|\frac{\partial m_e}{\partial T_1} \right|\Delta T_1 + \left|\frac{\partial m_e}{\partial T_e} \right|\Delta T_e
\end{equation}
Dalla Legge (11) e dalla Legge (29) otteniamo:
\begin{equation}
	m_e=(135.2\pm 1.9)\ g
\end{equation}



\subsection{Fase 2}
Per questa fase dell'esperienza è stata misurata una nuova massa d'acqua $m_1$.
\begin{equation}
	m_1=(201.09\pm 0.01)\ g
\end{equation}
Una volta versata la quantità d'acqua $m_1$ all'interno del calorimetro ne è stata misurata la temperatura $T_1$.
\begin{equation}
	T_1=(21.5\pm 0.1)\ ^{\circ}
\end{equation}
È stata misurata la massa $m_x$ del metallo.
\begin{equation}
	m_x=(46.99\pm 0.01)\ g
\end{equation}
Il metallo è stato poi immerso in acqua in stato di ebollizione per 10 minuti. In questo periodo di tempo il metallo si è portato alla temperatura $T_2$.
\begin{equation}
	T_2=(98.0\pm 0.1)\ ^{\circ}
\end{equation}
A questo punto il metallo è stato immerso rapidamente nel calorimetro e ne è stata misurata la temperatura d'equilibrio $T_e$.
\begin{equation}
	T_e=(24.0\pm 0.1)\ ^{\circ}
\end{equation}
Con queste misurazioni dirette è possibile stimare il calore specifico del metallo $c_x$ (tenendo conto che il calore specifico dell'acqua sia $c_a=1\frac{cal}{gK}$). La sua incertezza è stata valutata col metodo dei differenziali (Legge (17)).
\begin{equation}
	\Delta c_x=\left|\frac{\partial c_x}{\partial m_1} \right|\Delta m_1 + \left|\frac{\partial c_x}{\partial m_e} \right|\Delta m_e + \left|\frac{\partial c_x}{\partial m_x} \right|\Delta m_x + \left|\frac{\partial c_x}{\partial T_1} \right|\Delta T_1 + \left|\frac{\partial c_x}{\partial T_2} \right|\Delta T_2 + \left|\frac{\partial c_x}{\partial T_e} \right|\Delta T_e
\end{equation}
Dalla Legge (13) e dalla Legge (36) otteniamo:
\begin{equation}
	c_x=(0.24\pm 0.03)\ \frac{kcal}{kg\ K}
\end{equation}






