Scopo dell'esperienza è la stima del calore specifico di un metallo, utilizzando il calorimetro delle mescolanze, altrimenti noto come calorimetro di Regnault. Dal calore specifico misurato è stato poi determinato il suo materiale. L'esperienza si suddivide in due fasi:
\begin{itemize}
    \item Nella prima fase vine stimata la capacità termica del calorimetro, espressa in termini dell'equivalente in acqua. 
    \item Nella seconda fase viene stimato il calore specifico del metallo la cui composizione è ingota. In questa fase il metallo viene prima riscaldato e poi immerso nel calorimetro contenente acqua a temperatura ambiente.
\end{itemize}