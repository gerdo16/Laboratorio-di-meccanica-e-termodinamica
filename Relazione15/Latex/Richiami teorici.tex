Una lente convergente sottile è un sistema ottico in grado di deviare i raggi luminosi paralleli all'asse ottico verso un punto comune detto fuoco. 
La distanza tra il centro della lente e il fuoco è detta distanza focale $f$. Questa distanza rappresenta una delle caratteristiche fondamentali della lente e dipende 
sia dalla sua curvatura sia dal materiale con cui è costruita (indice di rifrazione). Nel caso di una lente sottile, la formazione dell'immagine può essere descritta dalla legge 
dei punti coniugati, espressa dalla seguente equazione:
\begin{equation}
    \frac{1}{p} + \frac{1}{q} = \frac{1}{f}
\end{equation}
dove:
\begin{itemize}
    \item $p$ è la distanza tra l'oggetto e la lente
    \item $q$ è la distanza tra l'immagine e la lente
    \item $f$ è la distanza focale della lente
\end{itemize}
Questa relazione permette di calcolare la distanza focale conoscendo le distanze oggetto-lente e immagine-lente. La formula è valida nel regime di ottica geometrica, 
quando le dimensioni della lente sono piccole rispetto alla distanza degli oggetti in gioco, e gli angoli dei raggi luminosi rispetto all'asse ottico sono piccoli 
(approssimazione parassiale). 
Nel laboratorio, la distanza focale può essere determinata sperimentalmente posizionando un oggetto luminoso e spostando uno schermo fino a ottenere un'immagine nitida.