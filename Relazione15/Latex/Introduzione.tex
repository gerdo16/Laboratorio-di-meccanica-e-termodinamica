Scopo dell'esperienza è la misura sperimentale della distanza focale di una lente convergente sottile. Il setup sperimentale è costituito da una guida su cui sono posizionati tre supporti: uno per ospitare una fonte luminosa, uno per uno schermo e un ultimo interposto tra gli altri due per ospitare la lente. Posizionando la fonte luminosa a distanze crescenti dalla lente, viene registrato un intervallo di distanze corrispondenti entro cui l'immagine sullo schermo appare nitida. Ripetendo la procedura, sono state raccolte 10 coppie di dati $(p, q)$, dove $p$ e $q$ rappresentano rispettivamente la distanza della fonte e quella dello schermo dalla lente. La legge che regola la fisica del fenomeno è la seguente:
\begin{equation}
    \frac{1}{p} + \frac{1}{q} = \frac{1}{f}
\end{equation}
dove $f$ è la distanza focale che si vuole stimare. Riportando in grafico i valori di $\frac{1}{q}$ in funzione di $\frac{1}{p}$ come segue:
\begin{equation}
    \frac{1}{q} = -\frac{1}{p} +  \frac{1}{f}
\end{equation}
è possibile stimare $f$ attraverso una retta dei minimi quadrati.